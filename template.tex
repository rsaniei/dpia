%%%%%%%%%%%%%%%%%%%%%%%%%%%%%%%%%%%%%%%%%%%%%%%%%%
%Copyright 2015 Idafen Santana, Mar�a Poveda
%
%Licensed under the Apache License, Version 2.0 (the "License");
%you may not use this file except in compliance with the License.
%You may obtain a copy of the License at
%
%    http://www.apache.org/licenses/LICENSE-2.0
%
%Unless required by applicable law or agreed to in writing, software
%distributed under the License is distributed on an "AS IS" BASIS,
%WITHOUT WARRANTIES OR CONDITIONS OF ANY KIND, either express or implied.
%See the License for the specific language governing permissions and
%limitations under the License.
%%%%%%%%%%%%%%%%%%%%%%%%%%%%%%%%%%%%%%%%%%%%%%%%%%

\documentclass{article}
\usepackage{graphicx}
\usepackage[table]{xcolor}
\usepackage{multirow}


\begin{document}

\title{ORSD template for \LaTeX{}}
\author{Idafen Santana \& Mar\'ia Poveda}
\maketitle

\begin{table}
\centering
\scriptsize
\begin{tabular}{| l | l | l | l  | l | l | l |l| }
\hline
\multicolumn{8}{|c|}{\cellcolor[HTML]{A0A0A0}\textbf{WICUS Ontology Requirements Specification Document}}                                                                                                                                                                                                                                                         \\ \hline
\multicolumn{8}{|c|}{\cellcolor[HTML]{EFEFEF}\textbf{1. Purpose}}                                                                                                                                                                                                                                                         \\ \hline
\multicolumn{8}{| p{14.0cm} |}{The purpose of this ontology is to represent entities involved in the DPIA to
support encoding of the DPIA-related obligations mandated by GDPR and different Data Protection Authorities (DPAs) using other Policy Expression Languages such as ODRL.}                                                                                                                                                                                                                                                      \\ \hline
\multicolumn{8}{|c|}{\cellcolor[HTML]{EFEFEF}\textbf{2. Scope}}                                                                                                                                                                                                                                                         \\ \hline
\multicolumn{8}{| p{14.0cm} |}{The scope of this ontology is limited to the definition of the entities necessary to represent pieces of the information related to the DPIA mentioned in the GDPR Article 35 as the main reference, along with the other guidelines and DPIA templates provided by the different Data Protection Authorities (DPAs). }

% necessary to validate the completeness of the whole DPIA process. This includes: \newline
% 1.To decide whether a DPIA is required for a specific processing operation.\newline
% 2.To decide whether a DPIA is complete in terms of the information it should provide to ensure compliance with the GDPR. }                                                                                                                                                                                                                                                      \\ \hline
\multicolumn{8}{|c|}{\cellcolor[HTML]{EFEFEF}\textbf{3. Implementation Language}}                                                                                                                                                                                                                                                         \\ \hline
\multicolumn{8}{| p{14.0cm} |}{OWL}                                                                                                                                                                                                                                                      \\ \hline
\multicolumn{8}{|c|}{\cellcolor[HTML]{EFEFEF}\textbf{4. Intended End-Users}}                                                                                                                                                                                                                                                         \\ \hline
\multicolumn{8}{| p{14.0cm} |}{
User 1. Compliance tools working based on the Semantic Technologies. \newline
User 2. Data controllers collecting personal data who needs to conduct DPIA. \newline
}                                                                                                                                                                                                                                                      \\ \hline
\multicolumn{8}{|c|}{\cellcolor[HTML]{EFEFEF}\textbf{5. Intended Uses}}                                                                                                                                                                                                                                                         \\ \hline
\multicolumn{8}{| p{14.0cm} |}{
Use 1. Representation of the pieces of information a DPIA should contain, for example, the nature, scope, and context of the processing activity, the state/status of the DPIA, etc. This representation supports describing DPIA-related rules. \newline
Use 2. Representation of the DPIA-related activities that data controllers should perform, for example, reviewing the DPIA, prior consultation with the DPAs, seeking the view of the Data Protection Officer (DPO), etc. This representation supports describing DPIA-related rules.\newline
% Use 2. Compliance validation: Having a machine-readable representation of the planned processing activity, the DPIA document, and the rules related to the DPIA, a compliance tool can reason over the provided information and assess compliance/violation. \newline
% Use 3. Advising  \newline
% Use 4. Description of the use...
}                                                                                                                                                                                                                                                      \\ \hline
\multicolumn{8}{|c|}{\cellcolor[HTML]{EFEFEF}\textbf{6. Ontology Requirements}}                                                                                                                                                                                                                                                         \\ \hline
\multicolumn{8}{|c|}{\cellcolor[HTML]{EFEFEF}\textbf{a. Non-Functional  Requirements}}                                                                                                                                                                                                                                                         \\ \hline
\multicolumn{8}{| p{14.0cm} |}{
NFR 1. The ontology will be made available online to the community following the FAIR principles.\newline  }                                                                                                                                                                                                                                                      \\ \hline
\multicolumn{8}{|c|}{\cellcolor[HTML]{EFEFEF}\textbf{b. Functional  Requirements: Groups of Competency Questions}}                                                                                                                                                                                                                                                         \\ \hline
\multicolumn{4}{|c|}{\cellcolor[HTML]{EFEFEF}CQG1. XXXX}                    & \multicolumn{4}{c|}{\cellcolor[HTML]{EFEFEF}CQG2. YYYY}    \\ \hline

\multicolumn{4}{ | m{6.8cm} |}{
CQ1. Competency question 1 \newline
CQ2. Competency question 2 \newline
CQ3. Competency question 3}	    &

\multicolumn{4}{ m{6.8cm} |}{
CQ4. Competency question 5 \newline
CQ5. Competency question 6 \newline
CQ6. Competency question 7}			\\ \hline

\multicolumn{8}{|c|}{\cellcolor[HTML]{EFEFEF}\textbf{7. Pre-Glossary of Terms}}                                                                                                                                                                                                                                                     \\ \hline
\multicolumn{8}{|c|}{\cellcolor[HTML]{EFEFEF}\textbf{a. Terms from Competency Questions + Frequency}}                                                                                                                                                                                                                                                     \\ \hline
\multicolumn{1}{ | m{3.15cm} | }{
Term1 Freq1 \newline
Term2 Freq2
}	&
\multicolumn{1}{ m{3.15cm} | }{
Term3 Freq3 \newline
Term4 Freq4
}      &
\multicolumn{1}{ m{3.15cm} | }{
Term5 Freq5 \newline
Term6 Freq6 } &
\multicolumn{5}{ m{3.15cm} | }{
Term7 Freq7 \newline
Term8 Freq8
}
\\ \hline
\multicolumn{8}{|c|}{\cellcolor[HTML]{EFEFEF}\textbf{b. Terms from Answers + Frequency}}                                                                                                                                                                                                                                                     \\ \hline
\multicolumn{1}{ | m{3.15cm} | }{
Term1 Freq1 \newline
Term2 Freq2
}	&
\multicolumn{1}{ m{3.15cm} | }{
Term3 Freq3 \newline
Term4 Freq4
}      &
\multicolumn{1}{ m{3.15cm} | }{
Term5 Freq5 \newline
Term6 Freq6 } &
\multicolumn{5}{ m{3.15cm} | }{
Term7 Freq7 \newline
Term8 Freq8
}
\\ \hline
\multicolumn{8}{|c|}{\cellcolor[HTML]{EFEFEF}\textbf{c. Objects}}                                                                                                                                                                                                                                                     \\ \hline
\multicolumn{8}{| p{14.0cm} |}{Object 1, Object 2, Object 3, Object 4, Object 5, ...}                                                                                                                                                                                                                                                      \\ \hline

\end{tabular}
\caption{Caption}
\label{tab:ORSD}
\vspace{-0.1in}
\end{table}

\end{document}